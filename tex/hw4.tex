\documentclass[11pt]{article}
\usepackage{fullpage}
\usepackage{fancyhdr}
\usepackage{epsfig}
\usepackage{algorithm}
\usepackage[noend]{algorithmic}
\usepackage{amsmath,amssymb,amsthm}
\usepackage{enumerate}


% FILL IN THE SPECIFICS OF EACH HOMEWORK HERE
\newcommand{\course}{CS 364}
\newcommand{\semester}{Spring 2013}
\newcommand{\name}{Artificial Intelligence}
%%%
%%%
%%% PLEASE FILL OUT YOUR NAME AND THE HWK NUMBER
%%%
%%%
\newcommand{\hwk}{Homework \#4 Solutions}
\newcommand{\student}{Oren Shoham, Peter Fogg, Sayer Rippey}

\newtheorem{lemma}{Lemma}
\newtheorem*{lem}{Lemma}
\newtheorem{definition}{Definition}
\newtheorem{notation}{Notation}
\newtheorem*{claim}{Claim}
\newtheorem*{fclaim}{False Claim}
\newtheorem{observation}{Observation}
\newtheorem{conjecture}[lemma]{Conjecture}
\newtheorem{theorem}[lemma]{Theorem}
\newtheorem{corollary}[lemma]{Corollary}
\newtheorem{proposition}[lemma]{Proposition}
\newtheorem*{rt}{Running Time}




%%% You can ignore the following stuff, it's just for formatting purposes
\textheight=8.6in
\setlength{\textwidth}{6.44in}
\addtolength{\headheight}{\baselineskip} 
% enumerate uses a), b), c), ...
\renewcommand{\labelenumi}{\alph{enumi})}
% Sets the style for fancy pages (namely, all but first page)
\pagestyle{fancy}
\fancyhf{}
\renewcommand{\headrulewidth}{0.0pt}
\renewcommand{\footrulewidth}{0.4pt}
% Changes style of plain pages (namely, the first page)
\fancypagestyle{plain}{
  \fancyhf{}
  \renewcommand\headrulewidth{0pt}
  \renewcommand\footrulewidth{0.4pt}
  \renewcommand{\headrule}{}
  }
% Changes the title box on the first page
\renewcommand\maketitle{
\begin{center}
\begin{tabular*}{6.44in}{l @{\extracolsep{\fill}}c r}
\bfseries  &  & \bfseries \course ~\semester \\
\bfseries&  & \bfseries  \hwk  \\
\bfseries   &   &  \bfseries \student \\ 
\end{tabular*}
\end{center} }




%%
%%
%% THE REAL STUFF STARTS HERE
%%
%%
\begin{document}
\maketitle
\thispagestyle{plain}


%%% PLEASE PLACE THE HONOR CODE AND YOUR NAME/SIGNATURE HERE
\noindent Honor Code: 

\subsection*{Part 1}
When we do leave-one-out cross validation, we run 50 trials, each time training the classifier on 49  examples and then testing on the 50th. If the 50th is from class A, then the classifier trains on 24 examples of class A and 25 examples of class B, and will predict B for the test. Likewise, if the 50th example is from class B, the classifier will predict A, because the majority of the training set is in class A. Therefore, for each of the 50 trials, the classifier will predict the wrong class and score zero.
\subsection*{Part 2}
\begin{itemize}
\item True. We encode every conjuction as a path from the root to a true leaf. We can split on an arbitrary attribute at any level of the tree. The tree must be complete, though, so we add a false node to any unused leaf nodes.
\item
\end{itemize}
\subsection*{Part 3}

\subsection*{Part 5}

{\bf Cybernetic Hands:} Scientists at the Federal University of Uberl$\hat{\text{a}}$ndia in Brazil developed an artificial neural network for controlling a prosthetic hand\footnote{Lamounier, E. D. G. A. R. D., et al. "A virtual prosthesis control based on neural networks for EMG pattern classification." Proceedings of the Artificial Intelligence and Soft Computing, Canada (2002).}. Myoletric protheses use electrical signals generated by the remaining muscles that used to control a missing limb to control the new, artificial limb. However, it has proven difficult for many patients to accurately control such prostheses on their own. \\
\\
Therefore, the authors of this paper designed a Multi-Layer Perceptron to classify electrical neuromuscular activity (known as EMG signals) from the upper arm into arm movements. The MLP used three layers: an input layer with either 4 or 10 neurons, depending on the way in which EMG input was processed, a hidden layer with 80 neurons, and an output layer with 4 neurons, each of which corresponded to a different type of arm movement. The MLP was trained using the backpropagation algorithm, and was able to accurately classify each type of arm movement with a 90\% or better success rate. \\
\\
The authors tested the neural network on a virtual simulation of a prosthetic, and the network was able to succesfully control the virtual limb. An artificial neural network has the advantage of being able to perform computations much faster than other types of prosthetic control systems. However, the authors have yet to test their system on a real prosthetic arm attached to a real person, so it remains to be seen whether their system will actually work in practice.

\end{document}
